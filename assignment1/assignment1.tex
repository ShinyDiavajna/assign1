%iffalse
\let\negmedspace\undefined
\let\negthickspace\undefined
\documentclass[journal,12pt,twocolumn]{IEEEtran}
\usepackage{cite}
\usepackage{amsmath,amssymb,amsfonts,amsthm}
\usepackage{algorithmic}
\usepackage{graphicx}
\usepackage{textcomp}
\usepackage{xcolor}
\usepackage{txfonts}
\usepackage{listings}
\usepackage{enumitem}
\usepackage{mathtools}
\usepackage{gensymb}
\usepackage{comment}
\usepackage[breaklinks=true]{hyperref}
\usepackage{tkz-euclide} 
\usepackage{listings}
\usepackage{gvv}                                        
%\def\inputGnumericTable{}                                 
\usepackage[latin1]{inputenc}                                
\usepackage{color}                                            
\usepackage{array}                                            
\usepackage{longtable}                                       
\usepackage{calc}                                             
\usepackage{multirow}                                         
\usepackage{hhline}                                           
\usepackage{ifthen}                                           
\usepackage{lscape}
\usepackage{tabularx}
\usepackage{array}
\usepackage{float}


\newtheorem{theorem}{Theorem}[section]
\newtheorem{problem}{Problem}
\newtheorem{proposition}{Proposition}[section]
\newtheorem{lemma}{Lemma}[section]
\newtheorem{corollary}[theorem]{Corollary}
\newtheorem{example}{Example}[section]
\newtheorem{definition}[problem]{Definition}
\newcommand{\BEQA}{\begin{eqnarray}}
\newcommand{\EEQA}{\end{eqnarray}}
\newcommand{\define}{\stackrel{\triangle}{=}}
\theoremstyle{remark}
\newtheorem{rem}{Remark}

% Marks the beginning of the document
\begin{document}
\bibliographystyle{IEEEtran}
\vspace{3cm}

\title{16.APPLICATIONS OF DERIVATIVES}
\author{EE24BTECH11058 - P.SHINY DIAVAJNA}

\maketitle
\newpage
\bigskip


\renewcommand{\thefigure}{\theenumi}
\renewcommand{\thetable}{\theenumi}

  Section-A JEE Advanced/IIT-JEE\\
  
  C.MCQ with One Correct Answer\\ 
    \begin{enumerate} 
      \item
	  If $f\brak{x}=x^3+bx^2+cx+d$ and $0<b^2<c,$ then in $(-\infty,\infty)$ \hfill(2004S)
        \begin{enumerate}
	 \item $f\brak{x}$ is a strictly increasing function
	 \item $f\brak{x}$ has a local maxima
	 \item $f\brak{x}$ is a strictly decreasing function
	 \item $f\brak{x}$ is bounded  \\
        \end{enumerate}
    
 
       \item
	       If $f\brak{x}=x^{\alpha} \log x$ and $f\brak{0}=0,$ then the value of $\alpha$ for which Rolles's theorem can be applied in $\sbrak{0,1}$ is 
		    \hfill(2004) 
        \begin {enumerate}
         \item $-2$
         \item $-1$
         \item $0$
         \item $1/2$\\
        \end{enumerate}
   
    
     \item
	     If $P\brak{x}$ is a polynomial of degree less than or equal to $2$ and $S$ is the set of all such polynomials so that $P\brak{0}=0,P\brak{1}=1$ and $P^{\prime}\brak{x}>0$ $\forall x \in \sbrak{0,1},$ then
     \hfill(2005S)
    \begin{enumerate}
        \item  $S=\phi$
        \item  $S=ax+(1-a)x^2$ $\forall a \in (0,2)$
        \item  $S=ax+(1-a)x^2$ $\forall a\in (0,\infty)$
        \item  $S=ax+(1-a)x^2$ $\forall a \in (0,1)$ \\
    \end{enumerate} 
    
   
   
      \item 
 The tangent to the curve $y=e^x$ drawn at the point $\brak{c,e^c}$intersects the line joining the points $\brak{c-1,e^{c-1}}$ and $\brak{c+1,e^{c+1}}$
    
      \hfill(2007-3 Marks)
      \begin{enumerate}
       \item on the left of $x=c$
       \item on the right of$x=c$ 
       \item at no point  
       \item at all points\\
      \end{enumerate}

   
      \item
	      Consider the two curves $C_{1}:y^2=4x,$  $C_{2}:x^2+y^2-6x+1=0.$ Then,  \hfill(2008)
     \begin{enumerate}
      \item $C_{1}$ and $C_{2}$ touch each other only at one point.
    
      \item $C_{1}$ and $C_{2}$ touch each other exactly at two points

      \item $C_{1}$ and $C_{2}$ intersect (but do not touch) at exactly two points

      \item $C_{1}$ and $C_{2}$ neither intersect nor touch each other \\
     
    \end{enumerate}
 
    \item 
    The total number of local maxima and local minima of the function \\
    \begin{align*}
      f\brak{x}=
      \begin{cases} 
       (2+x)^3 & \text{if } -3<x\le -1\\
       x^{2/3} & \text{if } -1<x<2
      \end{cases}
    \end{align*}is
   \hfill(2008)
    \begin{enumerate}
     \item $0$ 
     \item $1$
     \item $2$ 
     \item $3$ \\
    \end{enumerate}   


   \item 
	 Let the function $g:(-\infty,\infty) \rightarrow  \brak{-\frac{\pi}{2},\frac{\pi}{2}}$ be given by $g\brak{u}= 2 tan^{-1}\brak{e^u}-\frac{\pi}{2}$. Then,$g$ is  
   
   \hfill(2008)
   \begin{enumerate}

   \item even and is strictly increasing in ($0,\infty$)
 
   \item odd and is strictly decreasing in ($-\infty,\infty$)


   \item odd and is strictly increasing in ($-\infty,\infty$)

   \item neither even nor odd,but is strictly increasing in ($-\infty,\infty$) \\

   \end{enumerate}    

 
     \item 
	     The least value of a $\in$ $\mathbb{R}$ for which $4\alpha x^{2} + \frac{1}{x} \ge 1,$ for all $x>0,$ is 
     \hfill(JEE Adv. 2016)
     \begin{enumerate}
      \item $\frac{1}{64}$
      \item $\frac{1}{32}$
      \item $\frac{1}{27}$
      \item $\frac{1}{25}$\\
     \end{enumerate}
     
     \item
 If $f: R \rightarrow R$ is a twice differentiable function such that $f^{\prime\prime}\brak{x}>0$ for all $x \in R$ and $f\brak{\frac{1}{2}} = \brak{\frac{1}{2}},$ $f\brak{1}=1,$ then
     \hfill(JEE Adv. 2017) 

     \begin{enumerate}
	 \item $f^{\prime}\brak{1} \le 0$ 
	 \item $0<f^{\prime}\brak{1} \le \frac{1}{2}$
	 \item $\frac{1}{2} < f^{\prime}\brak{1}\le 1$ 
	 \item $f^{\prime}\brak{1} > 1$\\
    \end{enumerate}
  \end{enumerate}
 
    D. MCQs With One or More than One Correct
 \begin{enumerate}
     \item 
	     Let $P\brak{x} = a_0+ a_1x^2+a_2x^4+\dots a_nx^{2n}$ be a polynomial in a real variable $x$ with \\
		 $0<a_0<a_1<a_2<\dots a_n. $ The function $P\brak{x}$ has 
	 \hfill(1986- 2 Marks)
      \begin{enumerate}

         \item neither a maximum nor a minimum
 
         \item only one maximum

         \item only one minimum

         \item only one maximum and only one minimum
   
         \item none of these\\

      \end{enumerate} 

      \item
       If the line $ax+by+c = 0$ is a normal to the curve $xy=1,$ then 
	 \hfill(1986-2 Marks)
        \begin{enumerate}
         \item $a>0,b>0$ 
         \item $a>0,b<0$ 
         \item $a<0,b>0$
         \item $a<0,b<0$
         \item none of these.
       \end{enumerate}

      \item 
      The smallest positive root of the equation, $tanx-x=0$ lies in 
	\hfill(1987-2 Marks)
      \begin{enumerate}
       \item $\brak{0,\frac{\pi}{2}}$
       \item $\brak{\frac{\pi}{2},\pi}$
       \item $\brak{\pi,\frac{3\pi}{2}}$
       \item $\brak{\frac{3\pi}{2},2\pi}$
       \item None of these\\
      \end{enumerate}

        \item
	Let $f$ and $g$ be increasing and decreasing functions, respectively from $[0,\infty)$ to $[0,\infty)$. Let $h\brak{x} = f\brak{g\brak{x}}.$ If $h\brak{0} = 0,$ then $h\brak{x}-h\brak{1}$ is
	 \hfill(1987-2 Marks)
        \begin{enumerate}
          \item always zero
          \item always negative
          \item always positive
          \item strictly increasing
          \item None of these.\\
        \end{enumerate}

       \item 
       If 
	\begin{align*}
	 f\brak{x}=\begin{cases} 
	 3x^2+12x-1 & \text{if }-1 \le x\le 2\\
	 37-x & \text{if } 2<x \le 3 
         \end{cases}
       \end{align*}then:

	 \hfill(2008)
       \begin{enumerate}
	 \item $f\brak{x}$ is increasing on $\sbrak{-1,2}$
	 \item $f\brak{x}$ is continuous on $\sbrak{-1,3}$
	 \item $f^{\prime}\brak{2}$ does not exist
	 \item $f\brak{x}$ has the maximum value at $x=2$\\
       \end{enumerate}

     \item
	     Let $h\brak{x}$ = $f\brak{x}-(f\brak{x})^2+ (f\brak{x})^3$  for every real number $x.$ Then
     \hfill(1998-2 Marks)\\
     \begin{enumerate}
      \item $h$ is increasing whenever $f$ is increasing
      \item $h$ is increasing whenever $f$ is decreasing
      \item $h$ is decreasing whenever $f$ is decreasing
      \item nothing can be said in general.\\
     \end{enumerate}
 \end{enumerate}
\end{document}

